\documentclass{article}
\usepackage{amsmath}
\usepackage{amsfonts}
\usepackage{amssymb}
\usepackage{ctex}
\usepackage{graphicx}
\usepackage{float}
\usepackage{geometry}
\geometry{a4paper,scale=0.8}
\usepackage{caption}
\usepackage{subcaption}
\newcommand{\oiint}{\mathop{{\int\!\!\!\!\!\int}\mkern-21mu \bigcirc} {}}
\newcommand*{\dif}{\mathop{}\!\mathrm{d}}

\usepackage{parskip}
\setlength{\parindent}{0cm}


\author{胡喜平}
\title{理论力学复习}

\begin{document}
\maketitle

\section{达朗贝尔原理}

对于稳定的系统,有
\begin{equation}
  \sum\limits_{\alpha} \mathbf{F}_{\alpha} \cdot \delta \mathbf{r}_{\alpha} = 0
\end{equation}
将$\mathbf{F}_{\alpha}$分解为外力$\mathbf{F}_{\alpha}^{e}$和内力(约束力)$\mathbf{f}_{\alpha}$,其中
\begin{equation}
  \sum\limits_{\alpha} \mathbf{f}_{\alpha} \cdot \delta \mathbf{r}_{\alpha} = 0
\end{equation}
因此有
\begin{equation}
  \sum\limits_{\alpha} \mathbf{F}_{\alpha}^{e} \cdot \delta \mathbf{r}_{\alpha} = 0
\end{equation}
当问题不是静力学问题的时候,我们添加一个惯性力使之化为静力学问题,由于$\mathbf{F}_{\alpha} - \dot{\mathbf{p}}_{\alpha} = 0$
\begin{equation}
  \label{eq:0}
  \sum\limits_{\alpha}\left(\mathbf{F}_{\alpha}^{e} - \dot{\mathbf{p}}_{\alpha} \right) \cdot \delta \mathbf{r}_{\alpha} = 0
\end{equation}
这就是达朗贝尔准则

\section{拉格朗日方程}

\subsection{广义坐标}
位置$\mathbf{r}_{\alpha}$可以由坐标$q_{j}$表示
\begin{equation}
  \label{eq:5}
  \mathbf{r}_{\alpha} = \mathbf{r}_{\alpha} \left( q_{j},t \right)
\end{equation}
对它求导
\begin{equation}
  \label{eq:6}
  \dot{\mathbf{r}}_{\alpha} = \frac{\dif \mathbf{r}_{\alpha}}{\dif t} = \sum\limits_{j} \frac{\partial \mathbf{r}_{\alpha}}{\partial q_{j}} \dot{q_{j}} + \frac{\partial \mathbf{r}_{\alpha}}{\partial t}
\end{equation}
进而得到
\begin{equation}
  \label{eq:1}
  \delta \mathbf{r}_{\alpha}= \sum\limits_{j} \frac{\partial \mathbf{r}_{\alpha}}{\partial q_{j}} \delta q_{j}
\end{equation}

\subsection{拉格朗日方程的推导}
又方程\ref{eq:0}可以得到
\begin{equation}
  \label{eq:8}
  \sum\limits_{\alpha} \mathbf{F}_{\alpha}^{e} \cdot \delta \mathbf{r}_{\alpha} =  \sum\limits_{\alpha} \dot{\mathbf{p}}_{\alpha} \cdot \delta \mathbf{r}_{\alpha}
\end{equation}
对左边进行展开,代入\ref{eq:1}
\begin{equation}
  \label{eq:2}
  \sum\limits_{\alpha} \mathbf{F}_{\alpha}^{e} \cdot \delta \mathbf{r}_{\alpha} = \sum\limits_{\alpha} \mathbf{F}_{\alpha}^{e} \cdot  \sum\limits_{j} \frac{\partial \mathbf{r}_{\alpha}}{\partial q_{j}} \delta q_{j} = \sum\limits_{j} \sum\limits_{\alpha} \mathbf{F}_{\alpha}^{e} \cdot \frac{\partial \mathbf{r}_{\alpha}}{\partial q_{j}} \delta q_{j}
\end{equation}
定义广义力$Q_{j}$
\begin{equation}
  \label{eq:3}
  \sum\limits_{\alpha} \mathbf{F}_{\alpha}^{e} \cdot \frac{\partial \mathbf{r}_{\alpha}}{\partial q_{j}} = Q_{j}
\end{equation}
则式\ref{eq:8}左边可以化为
\begin{equation}
  \label{eq:11}
  \sum\limits_{\alpha} \mathbf{F}_{\alpha}^{e} \cdot \delta \mathbf{r}_{\alpha} = \sum\limits_{j} Q_{j} \delta q_{j}
\end{equation}
下面我们对右边进行展开,同样,代入\ref{eq:1}
\begin{equation}
  \label{eq:12}
  \sum\limits_{\alpha} \dot{\mathbf{p}}_{\alpha} \cdot \delta \mathbf{r}_{\alpha} = \sum\limits_{\alpha} \dot{\mathbf{p}}_{\alpha} \cdot \sum\limits_{j} \frac{\partial \mathbf{r}_{\alpha}}{\partial q_{j}} \delta q_{j} = \sum\limits_{\alpha}m_{\alpha} \ddot{\mathbf{r}}_{\alpha}  \cdot \sum\limits_{j} \frac{\partial \mathbf{r}_{\alpha}}{\partial q_{j}} \delta q_{j} =  \sum\limits_{\alpha} \sum\limits_{j} m_{\alpha} \ddot{\mathbf{r}}_{\alpha}  \cdot  \frac{\partial \mathbf{r}_{\alpha}}{\partial q_{j}} \delta q_{j}
\end{equation}
运用复合函数求导法则
\begin{equation}
  \label{eq:13}
  \frac{\dif}{\dif t} \left( m_{\alpha} \dot{\mathbf{r}}_{\alpha} \cdot \frac{\partial \mathbf{r}_{\alpha}}{\partial q_{j}} \right) = m_{\alpha} \ddot{\mathbf{r}}_{\alpha}  \cdot  \frac{\partial \mathbf{r}_{\alpha}}{\partial q_{j}} + m_{\alpha} \dot{\mathbf{r}}_{\alpha} \cdot \frac{\dif}{\dif t} \left(  \frac{\partial \mathbf{r}_{\alpha}}{\partial q_{j}}  \right)
\end{equation}
即
\begin{equation}
  \label{eq:14}
  m_{\alpha} \ddot{\mathbf{r}}_{\alpha}  \cdot  \frac{\partial \mathbf{r}_{\alpha}}{\partial q_{j}} = \frac{\dif}{\dif t} \left( m_{\alpha} \dot{\mathbf{r}}_{\alpha} \cdot \frac{\partial \mathbf{r}_{\alpha}}{\partial q_{j}} \right) -  m_{\alpha} \dot{\mathbf{r}}_{\alpha} \cdot \frac{\dif}{\dif t} \left(  \frac{\partial \mathbf{r}_{\alpha}}{\partial q_{j}}  \right)
\end{equation}
把\ref{eq:14}代入\ref{eq:12},得到
\begin{equation}
  \label{eq:15}
  \sum\limits_{\alpha} \dot{\mathbf{p}}_{\alpha} \cdot \delta \mathbf{r}_{\alpha} = \sum\limits_{\alpha} \sum\limits_{j} \left[ \frac{\dif}{\dif t} \left( m_{\alpha} \dot{\mathbf{r}}_{\alpha} \cdot \frac{\partial \mathbf{r}_{\alpha}}{\partial q_{j}} \right) -  m_{\alpha} \dot{\mathbf{r}}_{\alpha} \cdot \frac{\dif}{\dif t} \left(  \frac{\partial \mathbf{r}_{\alpha}}{\partial q_{j}}  \right) \right]\delta q_{j}
\end{equation}
因为
\begin{equation}
  \frac{\dif}{\dif t} \left(  \frac{\partial \mathbf{r}_{\alpha}}{\partial q_{j}} \right) = \frac{\partial \dot{\mathbf{r}}_{\alpha}}{\partial q_{j}}
\end{equation}
我们得到
\begin{equation}
  \label{eq:17}
  \sum\limits_{\alpha} \dot{\mathbf{p}}_{\alpha} \cdot \delta \mathbf{r}_{\alpha} = \sum\limits_{\alpha} \sum\limits_{j} \left[ \frac{\dif}{\dif t} \left( m_{\alpha} \dot{\mathbf{r}}_{\alpha} \cdot \frac{\partial \mathbf{r}_{\alpha}}{\partial q_{j}} \right) -  m_{\alpha} \dot{\mathbf{r}}_{\alpha} \cdot  \frac{\partial \dot{\mathbf{r}}_{\alpha}}{\partial q_{j}} \right]\delta q_{j}
\end{equation}
对\ref{eq:6}求偏导
\begin{equation}
  \label{eq:18}
  \frac{\partial \dot{\mathbf{r}}_{\alpha}}{\partial \dot{q_{j}}} = \frac{\partial \mathbf{r}_{\alpha}}{\partial q_{j}}
\end{equation}
带入\ref{eq:17}
\begin{equation}
  \label{eq:19}
  \sum\limits_{\alpha} \dot{\mathbf{p}}_{\alpha} \cdot \delta \mathbf{r}_{\alpha} = \sum\limits_{\alpha} \sum\limits_{j} \left[ \frac{\dif}{\dif t} \left( m_{\alpha} \dot{\mathbf{r}}_{\alpha} \cdot \frac{\partial \dot{\mathbf{r}}_{\alpha}}{\partial \dot{q_{j}}} \right) -  m_{\alpha} \dot{\mathbf{r}}_{\alpha} \cdot  \frac{\partial \dot{\mathbf{r}}_{\alpha}}{\partial q_{j}} \right]\delta q_{j}
\end{equation}
将\ref{eq:11}和\ref{eq:19}代入\ref{eq:8}
\begin{equation}
  \label{eq:20}
  \sum\limits_{\alpha} \sum\limits_{j} \left[ \frac{\dif}{\dif t} \left( m_{\alpha} \dot{\mathbf{r}}_{\alpha} \cdot \frac{\partial \dot{\mathbf{r}}_{\alpha}}{\partial \dot{q_{j}}} \right) -  m_{\alpha} \dot{\mathbf{r}}_{\alpha} \cdot  \frac{\partial \dot{\mathbf{r}}_{\alpha}}{\partial q_{j}} \right]\delta q_{j} = \sum\limits_{j} Q_{j} \delta q_{j}
\end{equation}
定义动能$T = \sum\limits_{\alpha} \frac{1}{2} m_{\alpha} \dot{\mathbf{r}}_{\alpha}^{2}$
\begin{equation}
  \label{eq:21}
  \partial T = \partial \left( \sum\limits_{\alpha} \frac{1}{2} m_{\alpha} \dot{\mathbf{r}}_{\alpha}^{2} \right) = \sum\limits_{\alpha} m_{\alpha} \dot{\mathbf{r}}_{\alpha} \cdot \partial \dot{\mathbf{r}}_{\alpha}
\end{equation}
将\ref{eq:21}代入\ref{eq:20}
\begin{equation}
  \label{eq:22}
  \sum\limits_{j} \left[ \frac{\dif}{\dif t} \left( \frac{\partial T}{\partial \dot{q_{j}}} \right) - \frac{\partial T}{\partial q_{j}} \right]\delta q_{j} = \sum\limits_{j} Q_{j} \delta q_{j}
\end{equation}
因此
\begin{equation}
  \label{eq:23}
  \frac{\dif}{\dif t} \left( \frac{\partial T}{\partial \dot{q_{j}}} \right) - \frac{\partial T}{\partial q_{j}} = Q_{j}
\end{equation}
即
\begin{equation}
  \label{eq:24}
  \frac{\dif}{\dif t} \left( \frac{\partial T}{\partial \dot{q_{j}}} \right) - \left( \frac{\partial T}{\partial q_{j}} + Q_{j} \right) = 0
\end{equation}
当$Q_{j}$是保守力,势能为$U$时
\begin{equation}
  \label{eq:25}
  -\frac{\partial U}{\partial q_{j}} = Q_{j}
\end{equation}
\begin{equation}
  \label{eq:26}
  \frac{\partial U}{\partial \dot{q_{j}}} = 0
\end{equation}
方程\ref{eq:24}可以化为
\begin{equation}
  \label{eq:27}
  \frac{\dif}{\dif t} \left( \frac{\partial (T-U)}{\partial \dot{q_{j}}} \right) - \frac{\partial (T-U)}{\partial q_{j}} = 0
\end{equation}
定义$L = T - U$,则
\begin{equation}
  \label{eq:28}
  \frac{\dif}{\dif t} \left( \frac{\partial L}{\partial \dot{q_{j}}} \right) - \frac{\partial L}{\partial q_{j}} = 0
\end{equation}

\section{哈密顿量的守恒}

对$L \left( q, \dot{q} ,t \right)$求导
\begin{equation}
  \label{eq:29}
  \frac{\dif L}{\dif t} = \sum\limits_{j} \frac{\partial L}{\partial q_{j}} \dot{q_{j}} + \sum\limits_{j} \frac{\partial L}{\partial \dot{q}_{j}} \ddot{q_{j}} + \frac{\partial L}{\partial t}
\end{equation}
式\ref{eq:28}告诉我们
\begin{equation}
  \label{eq:30}
  \frac{\dif}{\dif t} \left( \frac{\partial L}{\partial \dot{q_{j}}} \right) = \frac{\partial L}{\partial q_{j}}
\end{equation}
将\ref{eq:30}代入\ref{eq:29}
\begin{equation}
  \label{eq:31}
  \frac{\dif L}{\dif t} = \sum\limits_{j}\frac{\dif}{\dif t} \left( \frac{\partial L}{\partial \dot{q_{j}}} \right)  \dot{q_{j}} + \sum\limits_{j} \frac{\partial L}{\partial \dot{q}_{j}} \ddot{q_{j}} + \frac{\partial L}{\partial t}
\end{equation}
即
\begin{equation}
  \label{eq:32}
  \frac{\dif L}{\dif t} = \sum\limits_{j}\frac{\dif}{\dif t} \left( \frac{\partial L}{\partial \dot{q_{j}}} \dot{q_{j}} \right) + \frac{\partial L}{\partial t}
\end{equation}
交换求和与积分顺序
\begin{equation}
  \label{eq:33}
  \frac{\dif L}{\dif t} = \frac{\dif}{\dif t} \left( \sum\limits_{j}  \frac{\partial L}{\partial \dot{q_{j}}} \dot{q_{j}} \right) + \frac{\partial L}{\partial t}
\end{equation}
移项
\begin{equation}
  \label{eq:34}
  \frac{\dif}{\dif t} \left( \sum\limits_{j} \frac{\partial L}{\partial \dot{q_{j}}} \dot{q_{j}} -L \right) + \frac{\partial L}{\partial t} = 0
\end{equation}
定义系统的哈密顿量
\begin{equation}
  \label{eq:35}
  H = \sum\limits_{j} \frac{\partial L}{\partial \dot{q_{j}}} \dot{q_{j}} -L 
\end{equation}
当$\frac{\partial L}{\partial t}=0$时
\begin{equation}
  \label{eq:36}
  \frac{\dif H}{\dif t} = 0
\end{equation}
哈密顿量守恒

\section{Noether定理与广义动量}

由\ref{eq:33}
\begin{equation}
  \label{eq:37}
  \frac{\dif L}{\dif t} = \frac{\dif}{\dif t} \left( \sum\limits_{j}  \frac{\partial L}{\partial \dot{q_{j}}} \dot{q_{j}} \right) + \frac{\partial L}{\partial t}
\end{equation}
可以得到
\begin{equation}
  \label{eq:38}
  \delta L= \frac{\dif}{\dif t} \left( \sum\limits_{j}  \frac{\partial L}{\partial \dot{q_{j}}} \delta q_{j} \right)
\end{equation}
当空间具有平移不变性时,即虚位移的产生对于拉格朗日量没有影响时
\begin{equation}
  \label{eq:39}
  \delta L =0
\end{equation}
因此
\begin{equation}
  \label{eq:40}
  \frac{\dif}{\dif t} \left( \sum\limits_{j}  \frac{\partial L}{\partial \dot{q_{j}}} \delta q_{j} \right) = 0
\end{equation}
即
\begin{equation}
  \label{eq:41}
  \sum\limits_{j}  \frac{\partial L}{\partial \dot{q_{j}}} \delta q_{j} = \mathrm{const}
\end{equation}
因此定义广义动量
\begin{equation}
  \label{eq:42}
  p_{j} = \frac{\partial L}{\partial \dot{q_{j}}}
\end{equation}
则$p_{j} = \mathrm{const}$,广义动量守恒

\section{欧拉变分原理}

设
\begin{equation}
  \label{eq:43}
  J = \int_{x_{1}}^{x_{2}} f \left\{ y(x),y'(x);x  \right\} \dif x
\end{equation}
$J$是函数$y(x)$的函数,求$J$的极值函数,方法是另
\begin{equation}
  \label{eq:44}
  y \left( \alpha ,x \right) = y(x) + \alpha \eta (x)
\end{equation}
则
\begin{equation}
  \label{eq:45}
  y' \left( \alpha ,x \right) = y'(x) + \alpha \eta' (x)
\end{equation}
\begin{equation}
  \label{eq:46}
  \frac{\partial y}{\partial \alpha} = \eta (x)
\end{equation}
\begin{equation}
  \label{eq:47}
  \frac{\partial y'}{\partial \alpha} = \eta' (x)
\end{equation}
求解
\begin{equation}
  \label{eq:48}
  \frac{\partial J}{\partial \alpha} = 0
\end{equation}
解法为
\begin{equation}
  \label{eq:49}
  \frac{\partial J}{\partial \alpha} = \frac{\partial}{\partial \alpha} \int_{x_{1}}^{x_{2}} f \left\{ y(x),y'(x);x  \right\} \dif x =  \int_{x_{1}}^{x_{2}} \left( \frac{\partial f}{\partial y} \frac{\partial y}{\partial \alpha} + \frac{\partial f}{\partial y'} \frac{\partial y'}{\partial \alpha} \right) \dif x
\end{equation}
将\ref{eq:45}和\ref{eq:46}代入
\begin{equation}
  \label{eq:50}
  \frac{\partial J}{\partial \alpha} = \int_{x_{1}}^{x_{2}} \left( \frac{\partial f}{\partial y} \eta (x) + \frac{\partial f}{\partial y'} \eta' (x) \right) \dif x = \int_{x_{1}}^{x_{2}} \frac{\partial f}{\partial y} \eta (x) \dif x + \int_{x_{1}}^{x_{2}} \frac{\partial f}{\partial y'} \eta' (x) \dif x
\end{equation}
利用分步积分法算式\ref{eq:50}的最后一项
\begin{equation}
  \label{eq:51}
   \int_{x_{1}}^{x_{2}} \frac{\partial f}{\partial y'} \eta'(x) \dif x =  \int_{x_{1}}^{x_{2}} \frac{\partial f}{\partial y'} \dif \eta (x) = \frac{\partial f}{\partial y'} \left[ \eta (x_{1}) - \eta (x_{2}) \right] - \int_{x_{1}}^{x_{2}} \eta (x) \dif \left( \frac{\partial f}{\partial y'} \right)
 \end{equation}
 由于式\ref{eq:44}中$y(\alpha,x_{1}) = y(x_{1})$且$y(\alpha,x_{2}) = y(x_{2})$

 \begin{equation}
   \label{eq:52}
    \eta (x_{1}) = \eta (x_{2}) = 0
  \end{equation}
代入\ref{eq:51}
 \begin{equation}
   \label{eq:53}
   \int_{x_{1}}^{x_{2}} \frac{\partial f}{\partial y'} \eta'(x) \dif x = - \int_{x_{1}}^{x_{2}} \eta (x) \dif \left( \frac{\partial f}{\partial y'} \right) = - \int_{x_{1}}^{x_{2}} \eta (x) \frac{\dif}{\dif x} \left( \frac{\partial f}{\partial y'} \right) \dif x
 \end{equation}
 代入\ref{eq:50}
 \begin{equation}
   \label{eq:54}
   \frac{\partial J}{\partial \alpha} = \int_{x_{1}}^{x_{2}} \frac{\partial f}{\partial y} \eta (x) \dif x - \int_{x_{1}}^{x_{2}} \eta (x) \frac{\dif}{\dif x} \left( \frac{\partial f}{\partial y'} \right) \dif x = \int_{x_{1}}^{x_{2}} \left[ \frac{\partial f}{\partial y} -  \frac{\dif}{\dif x} \left( \frac{\partial f}{\partial y'} \right) \right]\eta (x) \dif x = 0
\end{equation}
 因此
\begin{equation}
   \label{eq:55}
   \frac{\partial f}{\partial y} -  \frac{\dif}{\dif x} \left( \frac{\partial f}{\partial y'} \right) = 0
 \end{equation}
 我们也可以用$\delta$记号表示,由式\ref{eq:54}和\ref{eq:46}
 \begin{equation}
   \label{eq:56}
    \frac{\partial J}{\partial \alpha} \dif \alpha = \int_{x_{1}}^{x_{2}} \left[ \frac{\partial f}{\partial y} -  \frac{\dif}{\dif x} \left( \frac{\partial f}{\partial y'} \right) \right] \frac{\partial y}{\partial \alpha} \dif \alpha \dif x = 0
  \end{equation}
  即
  \begin{equation}
    \label{eq:57}
    \delta J = \int_{x_{1}}^{x_{2}} \left[ \frac{\partial f}{\partial y} -  \frac{\dif}{\dif x} \left( \frac{\partial f}{\partial y'} \right) \right] \delta  y \dif x = 0
  \end{equation}

  \section{哈密顿原理}

  哈密顿原理的表示形式为
  \begin{equation}
    \label{eq:58}
    I = \int_{t_{1}}^{t_{2}} L \left( q_{j},\dot{q}_{j},t \right) \dif t= 0
  \end{equation}
  或
  \begin{equation}
    \label{eq:59}
    \delta I = \delta \int_{t_{1}}^{t_{2}} L \left( q_{j},\dot{q}_{j},t \right) \dif t= 0
  \end{equation}
  由式\ref{eq:55}我们已经知道,式\ref{eq:59}的解是
  \begin{equation}
    \label{eq:60}
    \frac{\partial L}{\partial q_{j}} -  \frac{\dif}{\dif t} \left( \frac{\partial L}{\partial \dot{q}_{j}} \right) = 0
  \end{equation}
  即拉格朗日方程
\end{document}